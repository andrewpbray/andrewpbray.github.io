\documentclass[10pt]{article}

\usepackage{amsmath,amssymb,amsthm}
\usepackage{fancyhdr,url,hyperref}
\usepackage{enumerate}

\oddsidemargin 0in  %0.5in
\topmargin     0in
\leftmargin    0in
\rightmargin   0in
\textheight    9in
\textwidth     6in %6in

\pagestyle{fancy}

\lhead{\textsc{MATH 141}}
\chead{\textsc{Practice}}
\lfoot{}
\cfoot{}
%\cfoot{\thepage}
\rfoot{}
\renewcommand{\headrulewidth}{0.2pt}
\renewcommand{\footrulewidth}{0.0pt}

\newcommand{\ans}{\vspace{0.25in}}
\newcommand{\R}{{\sf R}\xspace}
\newcommand{\cmd}[1]{\texttt{#1}}

\title{MATH 141:\\Intro to Probability and Statistics}
\author{Prof. Bray}
\date{Fall 2015}

\rhead{\textsc{September 21, 2014}}

\usepackage{Sweave}
\begin{document}
\Sconcordance{concordance:handout-04A-inference.tex:handout-04A-inference.Rnw:%
1 34 1 1 0 7 1 1 2 1 0 2 1 9 0 1 2 59 1}


\textbf{Is Yawning Contagious?} (statistical inference from first principles)  
Big idea: is there an association between two variables?  An experiment conducted by MythBusters tested if a person can be subconsciously influenced into yawning if another person near them yawns. You can watch a relevant portion of the episode here:  \url{http://dsc.discovery.com/tv-shows/mythbusters/videos/is-yawning-contagious-minimyth.htm}.

In this study 50 people were randomly assigned to two groups: 34 to a group where a person near them yawned (seeded) and 16 to a control group where there wasn't a yawn seed. The results are as follows:

\begin{Schunk}
\begin{Sinput}
> seeded <- c(rep(0, 12), rep(1, 24), rep(0, 4), rep(1, 10))
> yawned <- c(rep(0, 36), rep(1, 14))
> table(seeded, yawned)
\end{Sinput}
\begin{Soutput}
      yawned
seeded  0  1
     0 12  4
     1 24 10
\end{Soutput}
\end{Schunk}


\begin{enumerate}
  \item Here, what do you think is the explanatory variable?  Response variable? 
  \ans
  \item What is the probability of yawning, for the seeded group?
  \ans
  \item What is the probability of yawning, for the unseeded group?
  \ans
  \item If there were \emph{no association} between yawning and the proximity of another yawner, what would you expect the difference to be between these two probabilities?
  \ans
  \item Let $X$ be the number of people in the unseeded group that yawned.  What are the possible values for $X$?
  \ans
  \item In terms of $X$, what would be a more extreme result?  $X = $
  \ans
  \item What would another extreme result be?  $X = $
\end{enumerate}

\newpage

\textbf{Group activity (groups of 4)} Sampling from this table, assuming that there is no association between exposure to yawning and yawning yourself.  
\begin{enumerate}
  \item From your two decks, make one single deck that has 50 cards: 36 black and 14 red (yawners).  Set the extra cards aside.
  \item Shuffle deck well.
  \item Deal the deck out into two piles: one of 16 (unseeded) and one of 34 (seeded).
  \item Count up the number of red cards (yawners) in the pile of 16.
  \item Repeat steps 2-4, five times, taking turns.
  \item When your group is done, add your results to the board.
\end{enumerate}

\begin{center}
\begin{tabular}{|c|c|c|c|c|c|}
  \hline
  $x_1$ & $x_2$ & $x_3$ & $x_4$ & $x_5$ & $x_6$ \\
  \hline
  \hspace{0.75in} & \hspace{0.75in} & \hspace{0.75in} & \hspace{0.75in} & \hspace{0.75in} & \hspace{0.75in} \\[5ex]
  \hline
\end{tabular}
\end{center}

\vspace{2in}

\begin{center}
\begin{tabular}{|c|c|c|c|c|c|c|c|c|c|c|c|c|c|c|c|c|c|c|c|c|}
  \hline
  0 & 1 & 2 & 3 & 4 & 5 & 6 & 7 & 8 & 9 & 10 & 11 & 12 & 13 & 14 & 15 & 16 \\
\end{tabular}
\end{center}

\begin{enumerate}
  \item How many red cards would we expect (on average?)
  \ans
  \item What did we observe?
  \ans
  \item How would we summarize these results?   What is the big idea?
  \ans
\end{enumerate}


\end{document}
