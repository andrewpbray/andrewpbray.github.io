\documentclass[10pt]{article}

\usepackage{amsmath,amssymb,amsthm}
\usepackage{fancyhdr,url}
\usepackage{graphicx}

\oddsidemargin 0in  %0.5in
\topmargin     0in
\leftmargin    0in
\rightmargin   0in
\textheight    8.9in
\textwidth     6in %6in
%\headheight    0in
%\headsep       0in
%\footskip      0.5in

\newtheorem{thm}{Theorem}
\newtheorem{cor}[thm]{Corollary}
\newtheorem{obs}{Observation}
\newtheorem{lemma}{Lemma}
\newtheorem{claim}{Claim}
\newtheorem{definition}{Definition}
\newtheorem{question}{Question}
\newtheorem{answer}{Answer}
\newtheorem{problem}{Problem}
\newtheorem{solution}{Solution}
\newtheorem{conjecture}{Conjecture}

\pagestyle{fancy}

\lhead{\textsc{Prof. Bray}}
\chead{\textsc{Stat 340: Project Presentations}}
\rhead{\textsc{Fall 2014}}
\lfoot{}
\cfoot{\thepage}
\rfoot{}
\renewcommand{\headrulewidth}{0.2pt}
\renewcommand{\footrulewidth}{0.0pt}

\begin{document}

\section*{Project Presentations}

An effective oral presentation is an integral part of this project. One of the objectives of this class is to give you experience conveying the results of a technical investigation to a non-technical audience in a way that they can understand. Whether you choose to stay in academia or pursue a career in industry, the ability to communicate clearly is of paramount importance. As a data analyst, the burden of proof is on you to convince your audience that what you are saying is true. If your audience (who may very well be less knowledgeable about statistics than you are) cannot understand your results or their interpretations, then the technical merit of your project is irrelevant. 

	You will make a 15-minute oral presentation to the class. You should make (good) slides. Your goal should be to convey to your audience a clear understanding of your research question, along with a basic understanding of your model, and how well it addresses the research question you posed. You should \textbf{not} tell us everything that you did, or show a bunch of models that didn't work well. After hearing your talk, each student in the class should be able to answer: 
	\begin{enumerate}
		\item What was your project about? 
		\item What kind of model did you build? 
		\item How well did it work?  
		\item What did you learn from your model(s) about the topic at hand?
	\end{enumerate}

You should prepare electronic slides for your talk. PowerPoint, Google Presentation, Beamer (\LaTeX), or alternative, non-linear presentation software like Prezi are all fine. Use your creativity! One thing you should \emph{not} do is walk us through your calculations in RStudio. 
There will be an opportunity to rehearse your presentation with one of us a few days before your talk.

\paragraph{Advice}

There are many sources of advice for how to make a good presentation, but an excellent place to start is:
\begin{center}
	\url{http://techspeaking.denison.edu/}
\end{center}
Watch the videos on this site to identify some common mistakes. You should also read Joe Gallian's article on how to make a good presentation (\url{http://www.d.umn.edu/~jgallian/goodPPtalk.pdf}).

Here are is some general advice:
	\begin{itemize}
		\item Budget your time. You only have 15 minutes, and we will be running a very tight schedule. If your talk runs too short or too long, it makes you seem unprepared. Rehearse your talk ahead of time (with your group) several times in order to get a better feel for your timing. Note also that you may have a tendency to talk faster during your actual talk than you will during your rehearsal. Talking faster in order to speed up is not a good strategy -- you are much better off simply cutting material ahead of time. You will probably have a hard time getting through 15 slides in 15 minutes. 
		\item Don't write too much on each slide. You don't want people to have to read your slides, because if the audience is reading your slides, then they aren't listening to you. You want your slides to provide visual cues to the points that you are making -- not substitute for your spoken words. Concentrate on graphical displays and bullet-pointed lists of ideas. 
		\item Put your problem in context. Remember that most of your audience will have little or no knowledge of your subject matter. The easiest way to lose people is to dive right into technical details that require prior ``domain knowledge." Spend a few minutes at the beginning of your talk introducing your audience to the most basic aspects of your topic and present some motivation for what you are studying. 
		\item Speak loudly and clearly. Remember that you know more about your topic that anyone else in the room, so speak and act with confidence!
		\item Tell a story -- not necessarily the whole story. It is unrealistic to expect that you can tell your audience everything that you know about your topic in 15 minutes. You should strive to convey the big ideas in a clear fashion, but not dwell on the details. Your talk will be successful if your audience is able to walk away with an understanding of what your research question was, how you addressed it, and what the implications of your findings are. 
	\end{itemize}



\end{document}
